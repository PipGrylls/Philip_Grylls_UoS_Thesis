% Chapter 1

\chapter{Introduction} % Write in your own chapter title
\label{Chapter:Intro}
\lhead{Chapter 1. \emph{Introduction}} % Write in your own chapter title to set the page header

\section{Motivation}
\label{sec:Motivation}

\subsection{The answer we seek.}
%Why did people first start looking at galaxies
The firmament has been the muse of humans for as long as we have recorded our history and most likely longer still. The field of astronomy is descended from the priests who worshiped celestial objects as the divine and sought them to bring meaning to their world. Structures such as Stonehenge use celestial objects to allow people to track the repetitious passage of time, thus being able to predict the seasons. Humanity became so convinced that universe was there for our benefit, elaborate orrerys with the earth at the center of all creation were built to explain how the sun and planets orbit around us. The progression to modern astronomy was slow and instead of assuming the universe was built for humans we now try to discover our place amongst a cosmos that is unfathomably enormous, diverse and inhospitable. 

The Milky Way is our home and was the first observed galaxy, the term comes from Greek to mean `milky circle' stemming from our belief that the universe exists to give us life and gives the galaxy a mammalian nurturing characteristic. The idea of complexity emerging from the universe is still present in the first classification of the structures of galaxies by Edwin Hubble \citep{Hubble1926Extra-galacticNebulae.,Hubble1927TheNebulae}. Complex spirals were thought to be the descendents of elliptical galaxies leading to the misnomer of 'early type' (elliptical) and 'late type' (spiral) galaxies, as it has since been found that it is in fact spiral galaxies that transform into elliptical galaxies.

Most of the aforementioned advances in astronomy have come from technological advances. When the most advanced way to view the cosmos was by simply looking upward we knew little of what there was and interpreted as best we could. With the advent of optics, such as lenses and then mirrors, leading to the building of telescopes the ability to look deeper into the heavens become a possibility and the first work on classification was done. Modern advances such as CCD/SED photographic plates, and telescopes that work outside of the visible light spectrum information, give information far beyond what the human eye can see. The more information we gather the more theories of formation come to be, the emergent complexity of Hubble gives way to hierarchical formation destroying structure, galaxies are found to be the birthplace of stars like our sun and where some galaxies are forming many stars some appear dead, looking backwards in time astronomers find the cosmos appears to have passed the peak of creation with galaxies on average forming less stars now then ever before. The one thing that connects all astronomers over all ages is the desire to answer the ultimate question of `why?'.

%How has the study of galaxies progressed
\subsection{The tools we use.}
The field of astronomy is somewhat unique in science in that all our data and `experiments' already exist it is simply up to us to seek them out. Often the hypothesis follows the data in contrary to the scientific method of collecting data to support a hypothesis. Furthermore, for galactic astronomers our objects are effectively frozen in time with the timescales galaxies evolve on eclipsing not only a human lifespan but the entirety of human civilization. It is our good fortune that the `cosmic speed limit', the speed of light, means the further away from earth we look the further back in time and into the history of galaxy formation. By looking at galaxies at many different epochs of cosmic time we are able to construct a picture of how the galactic population has changed. Theories of how the early universe transforms to the universe we see locally emerged and a tools were required to test these theories that could show galactic formation on human timescales. The first such model by \citet{Holmberg1941OnProcedure.} predates digital computing using an array of light bulbs using the `candle power' to proxy mass and evolving the system using gravitational arguments and confirms the hypothesis that in distributed merging systems the loss of energy during a encounter results in a capture or merger as we know it today.

With the rapid increase in computing power over the last century, the capability of the simulation of galaxies has grown. More than simply testing the dynamics of mass, simulations now have prescriptions for the formation of gas from stars, the mergers of complex systems of tens of galaxies in a full cosmological setting, the feedback of energy from central black holes millions of times tha mass of our sun, and more... \citep[e.g.,][]{McAlpine2015TheCatalogues, Pillepich2018FirstGalaxies}. However advance these models have become a full understanding of our Universe still seems to beyond our capability. Other less computationally intensive tools have been used to try to understand from first principles the formation of our Universe mathematical analytics have been proposed to model galaxies. Since their original inception where galaxies form via the loss of angular momentum and cooling of gas to form galactic disks \cite{Mo1998TheDiscs}, these so called Semi-Analytic Models have branched out to cover tens of different analytics that try to balance different processes to faithfully recreate as many observable galaxy properties as possible \citep[e.g.,][]{DeLucia2006TheGalaxies, Guo2011FromCosmology}. The most recent development in galactic modelling is more humble in it approach, seeing not to recreate the entire universe but use what we observe as a guide. These Semi-Empirical Models constrain the model as much as possible using observations and then ask focused questions to understand if our hypothesis of formation adequately link the observed galactic population over cosmic history \citep{Hopkins2010MERGERSMATTER, Zavala2012, Moster2013, Shankar2014, Moster2018Emerge10}.

%What questions will we answer
\subsection{The questions we answer.}
In this thesis I describe \steel, the STastical sEmi-Empirical modeL, as my contribution to the galactic modelling community. 


\section{$\Lambda$CDM Cosmology}
\label{sec:LCDM}
%What is LCDM

%What are the predictions of LCDM

%Merger Trees and EPS formalism

%How do these predictions effect galaxy assembly

%The correlations found between DM and galaxies

\section{Hydrodynamic and N-Body Simulations}
\label{sec:Hydro}
%What is Hydrodynamics or N body

%State of the Art in N-Body DM modelling

%State of the art in Hydro Galaxy modelling

%Drawbacks of Hydro Modelling

\section{Semi-Analytic Modeling}
\label{sec:SAM}
%What is a SAM

%State of the art SAM

%Drawbacks of SAM

\section{Semi-Empirical Modelling}
\label{sec:SEM}
%What is a SEM

%State of the art SEM

%Drawbacks of SEM

\section{Galaxy Surveys Present and Future}
\label{sec:Surveys}
%First galaxy surveys

%SDSS

%COSMOS and CANDLES 

%Euclid


